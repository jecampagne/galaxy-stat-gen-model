\documentclass[11pt]{amsart}
\usepackage{amsaddr}
\usepackage{mathtools}
\usepackage{aas_macros} % macro pour Bibtex
\usepackage{hyperref}
\usepackage{graphicx} % Required for inserting images
\usepackage{amsthm}
\usepackage{amsmath}
% \usepackage{makecell}
\usepackage{amssymb}
\usepackage{enumitem}
\usepackage{color}
\usepackage{subcaption}
\usepackage{booktabs}
\usepackage{pifont}
\usepackage{natbib,fancyhdr} %new
%\usepackage{mathtools} % psmallmatrix

\newcommand{\cmark}{\ding{51}}%
\newcommand{\xmark}{\ding{55}}%

\setlength{\heavyrulewidth}{3\lightrulewidth}
\setlength{\abovetopsep}{1ex}

\usepackage[top=3cm, bottom=2cm, left=3cm, right=2cm]{geometry} %margins

% \usepackage{geometry}
%  \geometry{
%  a4paper,
%  left=25mm,
%  }

\newcommand{\jessa}[1]{{\color{red} #1}}
\newcommand{\field}[1]{\mathbf{#1}}
\newcommand{\nn}{\nonumber}


% My definitions:
\def\U{\mathcal U}
\def\LL{\mathcal L}
\def\P{\mathcal P}
\def\D{\mathcal D}

\DeclarePairedDelimiter\ceil{\lceil}{\rceil}
\DeclarePairedDelimiter\floor{\lfloor}{\rfloor}

% My packages:
\usepackage{algorithmic}
\usepackage[ruled,vlined,linesnumbered]{algorithm2e}
\usepackage{xcolor}
\usepackage{multicol} 
\usepackage{enumitem}
\newcommand\mycommfont[1]{\footnotesize\ttfamily\textcolor{blue}{#1}}
\SetCommentSty{mycommfont}
\newcommand{\bl}[1]{{\color{blue} #1}}
\newcommand{\rd}[1]{{\color{red} #1}}


% Alternative Assumption!

\newtheorem{theorem}{Theorem}
\newtheorem{assumption}[theorem]{Assumption}
\newtheorem{lemma}[theorem]{Lemma}
\newtheorem{corollary}[theorem]{Corollary}
\newtheorem{proposition}[theorem]{Proposition}
\newtheorem{fact}[theorem]{Fact}
\newtheorem{definition}{Definition}

% New environment
\newenvironment{assumptionp}[1]{
  \renewcommand\theassumptionalt{#1}
  \assumptionalt
}{\endassumptionalt}

\graphicspath{{./figures}}
\usepackage[margin=1cm]{caption}



\title{Galaxy image statistical generator: how to be confident?}
\author{Jean-Eric Campagne}
\address{Université Paris-Saclay, CNRS/IN2P3, IJCLab, 91405 Orsay, France
}
\email{jean-eric.campagne@ijclab.in2p3.fr}
\date{\today}

\begin{document}
\maketitle
%\renewcommand{\baselinestretch}{0.75}\normalsize
%\tableofcontents
%\renewcommand{\baselinestretch}{1.0}\normalsize
%
\begin{abstract}
blabla
\\
\smallskip
\noindent \textbf{Keywords.} diffusion models, U-Net
\end{abstract}

\section{Introduction}
\label{sec:Intro}
Image generation in Machine Learning is a challenging task that has made a dramatic rise in quality recently thanks to the large scale statistical model architectures as for instance \texttt{DALL-E 2} \citep{ramesh2022}, \texttt{Midjourney} \citep{Oppenlaender2022} and \texttt{StableDiffusion} \citep{Rombach2022} which use \textit{stochastic diffusion processes}. Such models have replaced the previous generation based on \textit{variational adversarial encoder} (aka VAE and GAN) as in \citep{KarrasALL18} or \textit{normalizing flows} such as in \texttt{Glow} \citep{Kingma2018}. Such models are generally trained using high resolution RGB images as \texttt{CelebA} dataset \citep{Liu2015} and \texttt{LSUN} bedroom dataset \citep{Yu2015}. 


The ability of such models have been rapidly adopted in many Physics domains. Concerning 
Astrophysics the generative models are used to create galaxy images of complex morphologies going beyond parametrized analytic light profile simulations (eg. \texttt{GALSIM} by \cite{ROWE2015121}). Along this line \cite{Lanusse2021} explore the use of generative models for instance to augment a galaxy image datasets \citep{ Fussell2019}, to improve deblending task \citep{Hemmati_2022,Arcelin2020}, or beeing used in a  Bayesian hierarchical pipeline providing a physical model for the point spread function and noise properties of individual observation and propose an hybrid VAE-Normalizing Flow architecture. Such generator can aid in cosmology the calibration of shape measurements for weak lensing in the context of large galaxy survey.



%\begin{figure}
%\centering
%\includegraphics[width=0.4\columnwidth]{filtreDb4.png}
%\caption{Coverage of the Fourier domain without gap by a low-pass filter and a collection of band-pass filters.}
%\label{fig-Daubechies-db4}
%\end{figure}


\section{Summary}


\section*{Acknowledgements}

\section*{Codes}


%%%%%%%%%%
\addcontentsline{toc}{section}{References}
% Put your bibiliography file here
%\section{Bibliography}
\bibliographystyle{elsarticle-harv} %aa
\bibliography{refs.bib}
%%%%%%%%%%%%%%%%%%%%%%%%%%%%%%%%%

\end{document}
